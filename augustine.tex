%!TEX TS-program = xelatex 
%!TEX TS-options = -synctex=1 -output-driver="xdvipdfmx -q -E"
%!TEX encoding = UTF-8 Unicode
%
%  augustine
%
%  Created by Mark Eli Kalderon on 2015-10-31.
%  Copyright (c) 2015. All rights reserved.
%

\documentclass[12pt]{article} 

% Definitions
\newcommand\mykeywords{Augustine, perception}
\newcommand\myauthor{Mark Eli Kalderon}

% Packages
\usepackage{geometry} \geometry{a4paper} 
\usepackage{url}
% \usepackage{txfonts}
\usepackage{color}
\usepackage{enumerate}
\definecolor{gray}{rgb}{0.459,0.438,0.471}
% \usepackage{setspace}
% \doublespace % Uncomment for doublespacing if necessary
% \usepackage{epigraph} % optional

% XeTeX
\usepackage[cm-default]{fontspec}
\usepackage{xltxtra,xunicode}
\defaultfontfeatures{Scale=MatchLowercase,Mapping=tex-text}
\setmainfont{Hoefler Text}

% Bibliography
\usepackage[round]{natbib}

% Title Information
\title{Trinitarian Perception}
\author{\myauthor} 
\date{} % Leave blank for no date, comment out for most recent date

% PDF Stuff
\usepackage[plainpages=false, pdfpagelabels, bookmarksnumbered, backref, pdftitle={Trinitarian perception}, pagebackref, pdfauthor={\myauthor}, pdfkeywords={\mykeywords}, xetex, colorlinks=true, citecolor=gray, linkcolor=gray, urlcolor=gray]{hyperref} 

%%% BEGIN DOCUMENT
\begin{document}

% Title Page
\maketitle
% \begin{abstract} % optional
% \noindent
% \end{abstract} 
% \vskip 2em \hrule height 0.4pt \vskip 2em
% \epigraph{Well, then, my friend, in the first place it is said that the earth, looked at from above, looks like those spherical balls made up of twelve pieces of leather; it is multi-colored, and of these colors those used by the painters give us an indication; up there the whole earth has these colors, but much brighter and purer than these; one part is sea-green and of marvelous beauty, another is golden, another is white, whiter than chalk or snow; the earth is composed also of the other colors, more numerous and beautiful than any we have seen.}{\textsc{Plato, \emph{Phaedo} 110 b--c}} % optional; make sure to uncomment \usepackage{epigraph}

% Layout Settings
\setlength{\parindent}{1em}

% Main Content

\section{Aporia} % (fold)
\label{sec:aporia}

Among the many pleasures of Greenwich Park are its ancient chestnut trees. In the early autumn, humans and squirrels vie with one another in foraging among the fallen burrs. Standing before one of the sweet chestnut trees replanted there when the park was redesigned for Charles \textsc{ii} in the 1660s, an organism of impressive size and age presents itself. The majority of its burrs remain on the tree and are brighter green than the surrounding foliage. It is early evening, and the light is long and golden. The light both articulates the fine texture of the bark and sets off the overall flow of the trunk in dramatic relief. Despite its manifest strength and solidity, the twisted trunk appears to be flowing in a wave like form. I come to realize that I am witnessing an organic process, the growth of the trunk, occurring so slowly as to appear, from within my limited temporal perspective, to be frozen, static. The difference in the scale of our lives is striking. For a moment, it induces in me a kind a kind of temporal vertigo.  Just as a radical difference in spatial scale can be vertiginous---think of how small one can feel when viewing the Milky Way---a radical difference in temporal scale can be vertiginous as well. The scale of its life and the strength manifest in centuries of growth make the sweet chestnut tree a fit object of awe. I find myself musing that in a different cultural context, perhaps one more prone to animism, the tree might reasonably be reckoned a god. 

I turn, and look, and see the ancient chestnut tree. In the natural history of my so seeing, the chestnut tree, the object of my perception, figures prominently as a causal antecedent. The tree's trunk appears to flow because the trunk has that flowing structure prior to its so appearing. And its flowing structure, illuminated in the golden light of the early evening, causes me to see that structure. This observation is perhaps the source of the long tradition of thinking of perception as a passive capacity. It is plausible, at the very least, that it is a materially necessary condition on a subject perceiving an object that the relevant sensory organs of the subject be acted upon, however mediately, by that object. Thus the chestnut tree must mediately alter my eyes---changing patterns of retinal stimulation, say---by altering the early evening's light in a complex subatomic process occurring at its surface.

I turn, and look, and see the ancient chestnut tree. The sweet chestnut tree mediately acts upon my eyes by acting upon the early evening illumination. And this is part of the natural history of my seeing the tree. Not only must my eyes be mediately acted upon by the tree, but my seeing the tree in the golden light with its bright green burs and flowing trunk is an experience that I undergo. Even though my seeing of the tree is an experience that I undergo as a result of the tree mediately acting upon my eyes, my seeing of the tree is not something \emph{done} to me by the tree. It is \emph{I} that see the tree. It is I, and not the tree, that is doing the seeing. Even acknowledging the passive elements in the natural history and phenomenology of my perception (that the tree is the cause of my seeing it and that my seeing it is an experience that I undergo), there is a residual active element in my seeing. The tree may make itself seen, but my seeing of the tree is activity properly attributed to me and not the tree.

How are we to understand this?

For counsel we turn to Augustine's account of perception in \emph{De Trinitate}. ``And God said, Let us make man in our image, after our likeness'' \emph{Genesis} 1:26. Noting God's use of the plural form, Augustine understands a human being to be an image and likeness of the Trinity, the Father, Son, and Holy Spirit. \emph{De Trinitate}, rather nicely, itself constitutes a trinity consisting in three parts. In the first part, Augustine reviews the biblical evidence for the doctrine of the Trinity. In the second part, Augustine explains how there may be three equal persons each identical to the whole of a substance in terms of his theory of relations. In the third part, Augustine applies these results to the human mind conceived as an \emph{imago Dei}. And it is in the context of the project of this third part, to understand the human mind as an image of the trinitarian God, that Augustine provides an account of perception.

\section{The Perceptual Trinity} % (fold)
\label{sec:the_perceptual_trinity}

\begin{quote}
	When we see a body, we have to consider and to distinguish the following three things, and this is a very simple task: first, the object which we see whether a stone, or a flame, or anything else that can be seen by the eyes, and this could of course, exist even before it was seen; secondly, the vision, which was not there before we perceived the object that was presented to the sense; thirdly, the power that fixes the sense of sight (\emph{sensus occulorum}) on the object that is seen as long as it is seen, namely, the attention of the mind (\emph{animi intentio}) (Augustine, \emph{De Trinitate}, 11 2 2; \citealt[61--62]{Matthews:2002ly})
\end{quote}

The perceptual trinity reappears later in chapter 2 in a more refined form:
\begin{quote}
	Since this is so, let us recall how these three, though differing in nature, may be fitted together into a kind of unity, namely, (i) the form of the body that is seen, (ii) its image impressed on the sense, which is vision, or the sense informed, and (iii) the will of the soul which directs the sense to the sensible thing and keeps the vision itself fixed upon it. (Augustine, \emph{De Trinitate}, 11 2 5; \citealt[65]{Matthews:2002ly})
\end{quote}

% section the_perceptual_trinity (end)

% section aporia (end)

\bibliographystyle{plainnat}
\bibliography{Philosophy}

\end{document}
