%!TEX TS-program = xelatex 
%!TEX TS-options = -synctex=1 -output-driver="xdvipdfmx -q -E"
%!TEX encoding = UTF-8 Unicode
%
%  augustine
%
%  Created by Mark Eli Kalderon on 2015-10-31.
%  Copyright (c) 2015. All rights reserved.
%

\documentclass[12pt]{article} 

% Definitions
\newcommand\mykeywords{Augustine, perception}
\newcommand\myauthor{Mark Eli Kalderon}
\newcommand\mytitle{Trinitarian Perception}

% Packages
\usepackage{geometry} \geometry{a4paper} 
\usepackage{url}
% \usepackage{txfonts}
\usepackage{color}
\usepackage{enumerate}
\definecolor{gray}{rgb}{0.459,0.438,0.471}
% \usepackage{setspace}
% \doublespace % Uncomment for doublespacing if necessary
% \usepackage{epigraph} % optional

% XeTeX
\usepackage[cm-default]{fontspec}
\usepackage{xltxtra,xunicode}
\defaultfontfeatures{Scale=MatchLowercase,Mapping=tex-text}
\setmainfont{Hoefler Text}

% Bibliography
% \usepackage[round]{natbib}

% Title Information
\title{\mytitle}
\author{\myauthor} 
\date{} % Leave blank for no date, comment out for most recent date

% PDF Stuff
\usepackage[plainpages=false, pdfpagelabels, bookmarksnumbered, backref, pdftitle={\mytitle}, pdfauthor={\myauthor}, pdfkeywords={\mykeywords}, xetex, colorlinks=true, citecolor=gray, linkcolor=gray, urlcolor=gray, unicode=true]{hyperref} 

%%% BEGIN DOCUMENT
\begin{document}

% Title Page
\maketitle
% \begin{abstract} % optional
% \noindent We begin with a puzzle about how to intelligibly combine the active and passive elements of perception. For counsel, we turn to Augustine's account of perception in \emph{De trinitate}. Augustine's trinitarian account of perception offers an attractive resolution of our puzzle. Augustine's resolution of our puzzle, however, cannot be straightforwardly adopted. It must be adapted. We end with speculation about how this might be done.
% \end{abstract}

% Layout Settings
\setlength{\parindent}{1em}

% Main Content

% Placed behind a protective grill facing the pavement on 111 Cannon Street resides the London Stone, or what remains of it. Composed of oolitic limestone, the London Stone is an unremarkable sight. The bronze plaque on the casing, however, ends on an enigmatic note, ``Its origin and purpose are unknown but in 1188 there was a reference to Henry, son of Eylwin de Lundenstane, subsequently Lord Mayor of London''. I am struck by the contrast between the unremarkable sight of the small block of limestone with its obscurity and continuing power over the imagination. And though these sentiments are somewhat in tension, it seemed at once poignant, the unimpressive degraded remains of a once great thing, and yet to intimate some unseen power, as if somehow some greatness remained, latent and dormant, within.

Walking down Cannon Street, I turn, and look, and see the London stone. In the natural history of my so seeing, the stone, the object of my perception, figures prominently as a causal antecedent. The stone presents a rough appearance because the stone has that surface prior to its so appearing. And its rough gray surface, illuminated from above by the light of its casing, causes me to see that surface. 
% This observation is perhaps the source of the long tradition of thinking of perception as a passive capacity. It is plausible, at the very least, that it is a materially necessary condition on a subject perceiving an object that the relevant sensory organs of the subject be acted upon, however mediately, by that object.
% Thus the London Stone must mediately alter my eyes---changing patterns of retinal stimulation, say---by altering the light in a complex subatomic process occurring at or near its surface.
% I turn, and look, and see the ancient stone.
% The London Stone mediately acts upon my eyes by acting upon the illumination.
Not only must my eyes be, at least, mediately acted upon by the stone, but my seeing the stone is an experience that I undergo. Even though my seeing of the stone is an experience that I undergo as a result of the stone mediately acting upon my eyes, my seeing of the stone is not something \emph{done} to me by the stone. It is \emph{I} that see the stone. It is I, and not the stone, that is doing the seeing. Even acknowledging the passive elements in the natural history and phenomenology of my perception (that the stone is the cause of my seeing it and that my seeing it is an experience that I undergo), there is a residual active element in my seeing. The stone may make itself seen, but my seeing of the stone is activity properly attributed to me and not the stone.

How are we to understand this?

For counsel we turn to Augustine. In \emph{De Trinitate}, Augustine distinguishes three elements in perception: (1) the object of perception, (2) perception, and (3) the attention of the mind (\emph{animi intentio}).

Let's consider these in turn:
\begin{enumerate}[(1)]
	\item The objects of at least visual perception include bodies and corporeal activities such as a stone or a flame. Augustine subscribes to a form of perceptual realism where the objects of perception may exist prior to perception. It is reasonable to suppose that this priority is both temporal and existential.
	\item The second element, ``vision'', should not be understood as sight, the power to see, but rather its exercise, seeing. What was absent before the perceiver saw the object was not their power to see. What was absent was not sight, but the seeing of the thing. According to Augustine, seeing or vision just is the sense in-formed by the perceived body or corporeal activity.
	\item So far, Augustine's enumeration of the elements of perception has been reasonably familiar. The real novelty of Augustine's trinitarian account is the insistence on the third essential element involved in perceptual apprehension, \emph{animi intentio}. \emph{Animi intentio} means something like concentration or conscious attention. Augustine evidently believes that perception involves consciously attending to the object of the perceptual experience. Moreover, \emph{animi intentio} is conceived by Augustine to be the activity of a cognitive power. Augustine conceives of this power as the will of the soul to direct the mind to various objects.
\end{enumerate}

% We have been asked to distinguish the object of perception from the perception of it, at least as a perceptual realist would understand these notions. 

Having enumerated the elements of the perceptual trinity, Augustine immediately notes an important disanalogy. The Father, Son, and Holy Spirit are, crucially, one substance. However, the perceptual trinity involves two substances: the object of perception understood as a body or corporeal activity independent of the perceiver and the living substance to whom the power of sight and its visual activity belong. The substance to which sense and vision belong, the living being, is a distinct substance from the material body or corporeal activity perceived, excepting, of course, when the perceiver sees a part of their own body, as when gazing at their hand. The disanalogy arises because perception is a trinitarian structure pertaining to the outer man and, strictly speaking only the inner man is an image of God. Nevertheless, the trinitarian structure of perception, while not itself an image of God, is like the Trinity, however imperfectly, for vestigial traces of the Trinity can be found in all created things

Consider the case of seeing a body such as a stone. Though the object of perception and the sense and vision differ in nature, since the object is an inanimate body and the perceptual power and its activity belong to the distinct substance of the living being, they are nevertheless united in perception. Like the Trinity, of which perception is a pale shadow, perception is a genuine unity. That a perceived body and the sense that unites with it in perception differ in nature suggests that their unity is subject to further explanation if genuinely intelligible. To get that explanation into view, we must understand how the elements of the perceptual trinity are arranged in ascending order of ontological dignity, and the explanatory relationships that obtain among them.

The object of perception, the stone, is a body. Moreover, it is an inanimate body. The sense that unites with the stone in the seeing of it, though bodily is animate and so not completely corporeal. Unlike the stone, the sense that unites with that inanimate body is a living body compounded of the corporeal sense organ and the sensitive soul that animates it. In contrast with the body and the sense that unites with it, the \emph{animi intentio} is completely incorporeal. \emph{Animi intentio} is the exercise of a power of the incorporeal soul. Thus the elements of the perceptual trinity are arranged in ascending order of ontological dignity, from the corporeal, to a compound of the corporeal and the spiritual, to the purely spiritual.

Assigning the elements of the perceptual trinity roles loosely analogous with the Father, Son, and Holy Spirit reveals explanatory relationships between them. The object of perception is the Father insofar as it begets its perception. Perception, then, would be the Son, leaving \emph{animi intentio} as the Holy Spirit. However, as soon as he makes the crucial observation that determines the assignment of trinitarian roles, Augustine, importantly, qualifies it. It turns out that this qualification is the key to understanding the unity manifest by the perceptual trinity despite their difference in nature. The object of perception, the form of the body, is, at best, a quasi-parent and the perception, the subsequent vision, a quasi-offspring. Part of the point of the qualification is that the object of perception is not sufficient for perception. The sense may assimilate to the form of the body in vision. That is part of what licenses us to speak of the body as a quasi-parent and the vision as quasi-offspring. But the body is causally insufficient for the occurence of vision. Augustine's idea is that in order for the form of the body to impress the sense, the sense must first be applied to the body. \emph{Animi intentio} fixing the sense to the body is a necessary precondition for the sense to receive an image and likeness of the form of the body in vision. 
% That is the function of the Holy Spirit in the guise of \emph{animi intentio}. 

Augustine describes \emph{animi intentio} as the will of the soul that directs the sense to the visible body.
 % \emph{Animi intentio} plays at least three distinct roles in the \emph{De trinitate} account. First, the will of the soul directs the sense outwards toward a distal body. That is, \emph{animi intentio} is the voluntary application of the visual sense to this distal body rather than some other. Or perhaps it voluntarily directs the sense at no thing in particular willing only to see what there is to see. It is thus operative prior to the sense's assimilation of the form of the body in vision. Second, \emph{animi intentio} unites the sense with the body in order to see it. It thus fixes the content of perception by directing the sense upon the object of perception. But having fixed the content of perception by directing the sense upon the object of perception, it may not safely withdraw from the scene. \emph{Animi intentio} is thus required, thirdly, to sustain that perception---it ``keeps the vision itself fixed upon it''.
The form of the body is a quasi-parent since it may only in-form the sense by the sense being applied to it by \emph{animi intentio}. In perceiving a body, the body and sense are bound together. As a consequence of their being bound together the sense formally assimilates to the form of the body and forms an image and likeness of it. What binds the body and sense together despite their differing in nature---while the former is a body the latter is a compound of body and soul---is the spiritual activity of the will of the soul that directs the sense to that body and so binds them. It is the will's binding the sense to the body that constitutes the unity of the perceptual trinity.

There are weaker and stronger interpretations of Augustine's here. 

The weak interpretation begins with the observation that being causally insufficient for perception at best entails that the object of perception is not the total cause of perception. But even if the object of perception were not the total cause, nor even the primary agent of the causal process, it causally contributes to the in-forming of sense.

The strong interpretation denies that the corporeal object of perception makes any causal contribution whatsoever. This seems to be Augustine's position in \emph{Genesis literally interpreted} where he claims that it is not the body that makes the image in the soul but the soul that makes the image in its own substance. At the very least, the soul is both the efficient and material cause of the image of the perceived body.

Bracketing the proper interpretation of \emph{De Trinitate}, the weak interpretation not only provides a solution to our initial puzzle but avoids a philosophical problem facing the strong interpretation.

The philosophical problem is due to Duns Scotus in the \emph{Ordinatio}. If the soul makes the image in itself and from itself, what explains that the soul, in so doing, formally assimilates to the object. Scotus' idea is that the demands of formally assimilating to the object requires that the object play some explanatory role in in-forming the sense. Otherwise, the form of the corporeal object must somehow determine the content of perception without itself being a determinant. But how could this be?

The weaker interpretation avoids this difficulty. Though the corporeal object of perception is not the total cause of the perception, nor even the primary agent of the causal process, it may, nevertheless, causally contribute to the in-forming of the sense, and so determining the content of perception.

Recall, our initial puzzle concerned how to intelligibly combine the active and passive elements in perception. The object of perception is a causal antecedent and the perceptual experience something the perceiver undergoes, and yet perceiving is something that the perceiver does and not something done to the perceiver. Augustine’s trinitarian account is well-placed to vindicate the residual active element in perception. That the will of the perceiver’s soul binds the body and the sense together so that an image and likeness of the former arises in the latter is something that the perceiver does. The activity of \emph{animi intentio} is properly attributed to the perceiver. It is not the stone that directs the sense upon it but the will of the soul of the perceiver. Not only is \emph{animi intentio} properly attributable to the perceiver, but it also constitutes the unity of the perceptual trinity. The role that Augustine assigns to the will in perception thus provides a straightforward sense in which perception is something the perceiver does as opposed to something done to the perceiver by the object of perception.

The Augustinian solution to our puzzle is attractive. Unfortunately, it cannot be straightforwardly be adopted. Can it somehow be adapted? We begin by dropping Augustine’s dualism. I do not mean to be embracing, instead, monism, whether of a neutral, reductive, or anomalous variety, so much as suspending judgment about the issue. One effect of this is that we may not rely on the principle that prohibit ontological inferiors from acting upon ontological superiors. 
% And the present essay, being composed at what Dr Thomas Arnold, headmaster of Rugby, described as “the godless institution in Gower Street”, will, needless to say, make nothing of the theological presuppositions of Augustine’s discussion. I thus align myself with the philosophical radicals, as opposed to the noncoformists, among the founders of the University of London.
Another departure concerns Augustine notion of \emph{animi intentio}. That the notion be adapted is dictated by our dropping Augustinian dualism. For, having done so, we may no longer conceive of it as the exercise of a power of an inextended, incorporeal soul. A question thus remains about whether anything may be found to play the role of \emph{animi intentio} well enough to sustain the Augustinian solution to our \emph{aporia}.

The first two elements of the perceptual trinity are easily accommodated. Let the objects of perception be existentially and temporally prior to the act of perception, and let's not insist that they be corporeal if we are of a mind to reckon shadows, holes, and rainbows among \emph{visibilia}. And let's understand perceptual assimilation as the determination, at least in part, of the phenomenological character of perceptual experience by the visible character of its object, relative to the perceiver’s partial perspective in the circumstances of perception. So understood, the image and likeness of the form of the object is less an inner representation than the phenomenological character of the experience determined, in part, by the object’s sensible form.

What about the third element, \emph{animi intentio}? Looking, like the will’s direction of the sense, is an activity. Moreover, looking, like \emph{animi intentio}, is an activity that is directed outwards. Looking is a voluntary intentional action that directs the perceiver’s visual awareness to distal aspects of the natural environment. When, on Cannon Street, I turn, and look, and see the London Stone the activity of looking was operative prior to the stone coming into view, just as \emph{animi intentio} is operative prior to binding the sense to the body so that the former may assimilate to the latter. And just as \emph{animi intentio} fixes the sense to the body, looking at the London Stone fixes the explicit awareness afforded me by my perceptual experience of that stone, so that it is the conscious object of attention. Not only does looking at the stone initiate my seeing of that stone, but looking sustains my seeing of it as well. Should I look away, I would cease to see the stone. So it would seem, then, that looking plays three roles analogous to \emph{animi intentio} in the trinitarian account. 

The confluence is potentially misleading, however. On Augustine’s understanding of \emph{animi intentio} and how it fulfils these roles, this is sufficient to understand how the will of the soul constitutes the unity of the perceptual trinity. And yet looking may have the features described consistent with not playing a constitutive role in the visual presentation of its object. Augustine’s Christianised Platonism is doing some work in his account of perception. That \emph{animi intentio} binds the sense to the body is a manifestation of its ontological superiority in being completely incorporeal. Having dropped Augustine’s dualism debars us from making any similar claim. The difficulty is that \emph{animi intentio}’s constituting the unity of the perceptual trinity was part of what licensed us to judge that perception was something that the perceiver does, as opposed to something done to the perceiver. Can we find a constitutive role for looking in the visual presentation of its object?

Concerning this explanatory lacuna I offered two speculative remarks. These are substantiated, if at all, in a forthcoming monograph from Cambridge University Press.

First in looking at an opaque body we experience a limit to our perceptual activity. Being opaque, we can neither see in or through that body. Perhaps an experienced limit to looking plays a constitutive role in seeing. 

An immediate problems prompts the second speculative remark as remedy. Not all limits to perceptual activities are external, such as opaque bodies that resist our gaze. Others are internal. Thus some perceivers can see further than I can. That my gaze can only extend so far is an internal limitation on my perceptual activity. 

If the visual presentation of the perceptually impenetrable is due to the operation of sympathy, then we have the basis of an answer. It is only when I experience the stone’s limit to my visual activity, its resistance to my gaze, its perceptual impenetrability, as a sympathetic response to a countervailing force, my gaze encountering an alien force that resists it, one force in conflict with another, like it yet distinct from it, that the perceptually impenetrable body discloses itself to visual awareness.

So consider the perceptual situation as conceived by Augustine. It is a unified manifold, where one part of the manifold, the perceiver's experience, formally assimilates to another part of the manifold, the perceived object, and does so because the parts disposed as they are are united in the manifold in the way that they are. According to Plotinus, sympathy just is the principle governing such cases of formal assimilation, be they cases of fellow-feeling or perception. 

The speculative suggestion, then, is that looking may play a constitutive role
in the visual presentation of its object, if visual presentation is understood to be a sympathetic reaction to an experienced limit to the perceiver’s activity. Insofar as visual presentation is governed by the principle of sympathy, it is a mode of being-with. Visual presentation is a mode of being-with insofar as it unites potentially distinct substances, the object of perception and the perceiver whose experience presents it. Turning, and looking, and seeing the London Stone is a way of being with that stone. Visual presentation is a way of being with that which resists our gaze.


\end{document}